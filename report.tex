\documentclass[12pt]{article}
\usepackage{amsmath}
\usepackage{amssymb}
\usepackage{graphicx}
\usepackage{comment}
\usepackage{longtable}
\usepackage{graphicx}

\begin{document}

\title{\textbf{Suduko Solver}}
\maketitle

\begin{center}
\title{\textbf{Advanced Analysis and Algorithms COMS3005}}
\maketitle 
\end{center}
\begin{center}
\title{\textbf{Report}}
\maketitle 
\end{center}

\begin{center}
Completed by:\\
Jared Naidoo - 719238
\\Krupa Prag - 782681
\\Zayyan Variawar - 852486
\end{center}

\newpage

\tableofcontents

\newpage

\section{Introduction}
\subsection{Aim}
\begin{flushleft}
The scenario at hand is to implement an backtracking algorithm to solve a Suduko Puzzle. The aim of this assignment is to gain insight into the concept in measuring performanceof an algorithm and relating these  measurements to the theoratical analysis of the algorithm, through this exploring the nature of Computer Science. 
\end{flushleft}
\subsection{Objective}
\begin{flushleft}
The objective of achieving the aim of this assignment, is to implement an backtracking algorithm in order to solve a partially filled Suduko Puzzle grid, with a unique solution. \\
Using the results of the implementation, from our experience and theoratical knowledge, we can give some insight on the performance of the algorithm that is executed. 
\end{flushleft}	

\subsection{Background and History}
\begin{flushleft}
Number Place - is the original name for a Sudoku Puzzle.\\
A Sudoku Puzzle is a logic based puzzle filled with various combinations of numbers. The objective of the puzzle is to have nine 3X3 grid filled with the number 1 to 9, without repitition in the sub grid, it's row and it's respective column of position, outputting a unique solution. \\
This particular number puzzle had made it' first appearance in the late 19th century's French newspapers, when puzzle setters begun amending and manipulating numbers in the magic squares. 
\end{flushleft}


\begin{center}
\textbf{\textit{"An expression mean nothing to me, unless it expresses a thought of God"} \\ - Srinivasa Ramanujan}
\end{center}


\begin{flushleft}
The roots of a Sudoku Puzzle is the Magic Square, which is particularly seen as a powerful means of representing combinatoral numbers in various respects of science. Great ancient Indian mathematician, Varahamihira, used a fourth-order magic square to specify recipes for making perfumes. Even as early as (ca. 900 A.D.), in Siddhayoga, displays the oldest dated third-ordered magic square, representing Vrnda's medical work on means to ease childbirth.
\end{flushleft}

\section{Analysis and Algorithm}
\subsection{Backtracking Algorithm}
\section{Implementation}
\section{Results}
\section{Conclusion}
\section{Glossary}
\begin{itemize}
\item Algorithm: 
\item Suduko Puzzle : a puzzle in which players insert the numbers one to nine into a grid consisting of nine squares subdivided into a further nine smaller squares in such a way that every number appears once in each horizontal line, vertical line, and square.
\item Suduko grid:  a puzzle in which several numbers are to be filled into a 9x9 grid of squares so that every row, every column, and every 3x3 box contains the numbers 1 through 9
\item: Magic squares: a square that is divided into smaller squares, each containing a number, such that the figures in each vertical, horizontal, and diagonal row add up to the same value

\end{itemize}
\section{Appendix}
\subsection{Dataset}
\subsection{Results}
\section{References}
%\begin{itemize}
%\item https://www.google.com/search?client=ubuntu&channel=fs&q=Magic+square+&ie=utf-8&oe=utf-8 (2016-09-23)
%\item https://www.google.com/search?client=ubuntu&channel=fs&q=Magic+square+&ie=utf-8&oe=utf-8#channel=fs&q=sudoku+definition (2016-09-23)
%\item http://timesofindia.indiatimes.com/home/sunday-times/What-is-the-history-of-Sudoku/articleshow/9169332.cms (2016-09-23)
%\item https://illuminations.nctm.org/Lesson.aspx?id=655 (2016-09-23)
%\end{itemize}
\section{Group contribution}
\end{document}